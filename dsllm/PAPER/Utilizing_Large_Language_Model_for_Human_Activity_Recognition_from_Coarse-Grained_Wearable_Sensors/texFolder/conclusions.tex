\chapter{Conclusion}

\hspace{2em}This thesis presents a novel approach to power-efficient sleep stage classification designed for consumer wearable devices. The primary focus has been on developing a deep learning model that balances computational efficiency with classification accuracy, making it suitable for deployment in resource-constrained environments.

The research explored the integration of 3-axis motion data, heart rate, and temporal information relative to light-off as input features. A key contribution of this work is the hierarchical model architecture, which employs a combination of simple and deep classifiers. This architecture allows the model to handle simple classification tasks with minimal computational resources, while reserving more complex feature extraction and classification for a deep classifier. By introducing a confidence-based decision mechanism, the model can dynamically determine whether further processing is necessary, thus optimizing both performance and efficiency.

The implementation of confidence thresholds has proven to be effective in managing the trade-off between computational cost and classification performance. A threshold of 0.8 was identified as providing an optimal balance, improving F1 score and Cohen's kappa while maintaining a low computational footprint. Additionally, the use of multiple simple classifiers at various stages of the CNNTransformer architecture demonstrated the benefits of hierarchical design in enhancing model performance for more challenging tasks.

The performance comparison between the proposed method and the CNNTransformer baseline highlighted significant reductions in computational cost—by approximately 80\%—and in the number of parameters—by nearly 90\%—on the SleepAccel dataset. While there was a slight trade-off in accuracy, the hierarchical design and confidence-based mechanism led to notable improvements in precision, recall, and overall classification reliability.

In conclusion, the methods developed in this thesis offer a viable solution for power-efficient sleep stage classification on consumer wearable devices. By leveraging hierarchical structures, parameter sharing, and confidence-based decision making, the proposed approach effectively balances the demands of computational efficiency and classification accuracy, paving the way for more practical and scalable sleep monitoring solutions.
