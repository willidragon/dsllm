人類行為辨識(HAR)是穿戴式感測技術中的重要研究領域,對於健康監測、運動分析和智慧生活應用具有重要意義。傳統的穿戴式感測器往往受限於低精度數據和計算資源限制,難以實現準確的行為辨識。隨著大型語言模型(LLM)的快速發展,其強大的序列理解和模式識別能力為解決此問題提供了新的可能。

本論文提出了一種利用大型語言模型進行低精度穿戴式感測行為辨識的創新方法。我們設計了一個適應性框架,將穿戴式感測器的粗粒度時間序列數據轉換為語言模型可理解的表示形式,並利用預訓練語言模型的強大表示學習能力來提升行為辨識的準確性。

我們的方法包含三個核心組件:(1)感測數據預處理和特徵提取模組,將原始感測信號轉換為結構化序列;(2)基於大型語言模型的行為模式學習架構,利用注意力機制捕捉時序關係;(3)多模態融合策略,整合不同感測器的信息以提升辨識性能。實驗結果顯示,相比傳統方法,我們的方法在多個公開數據集上都實現了顯著的性能提升,特別是在處理低精度和噪聲數據方面表現優異。

這項研究為穿戴式設備的智能化發展提供了新的技術路徑,推動了大型語言模型在物聯網和普適計算領域的應用創新。

關鍵字:人類行為辨識;大型語言模型;穿戴式感測器;低精度數據;時間序列分析;多模態融合